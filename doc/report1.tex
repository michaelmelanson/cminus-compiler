\documentclass[12pt]{amsart}
\usepackage{fullpage}

\title{CIS*4650 Checkpoint 1 report}
\author{Michael Melanson}

\begin{document}
\maketitle

\section{Summary}
The goal of this checkpoint was to implement the tokenization and
syntactic analysis phases of a C-Minus compiler. That is, the program
should input a character stream from disc, and output its equivalent
abstract syntax tree.


\section{Building and running}

\subsection{Dependencies}
As described in Section~\ref{implementation}, the scanner and parser
are written in Haskell. Since a Haskell compiler is not standard on
most operating systems, the \verb|ghc| (Glasgow Haskell Compiler)
package will have to be installed. In addition, the Parsec ``parser
combinator library'' is required.

On Ubuntu or Debian, this can be done by running:
\begin{verse}
\verb|sudo apt-get install ghc haskell-parsec|
\end{verse}

On Fedora, CentOS, or other RPM-based distributions, this can be done
by running\footnote{I do not have access to such a system, so this is
  an educated guess; my understanding is that Parsec is included with
  GHC.}:
\begin{verse}
\verb|sudo yum install ghc|
\end{verse}

Other operating systems will have similar installation procedures.

\subsection{Building}
Internally, the Haskell build system Cabal is used to automate
building and packaging of the code. However, as per the assigment
specification, a Makefile has been included; this Makefile simply
invokes Cabal. The Makefile includes the following rules:

\begin{description}
\item[all] The default rule; invokes \emph{configure} then
  \emph{build}. 
\item[configure] Checks dependencies, prepares the \verb|dist| directory.
\item[build] Compiles and links the executable files.
\item[clean] Removes all generated files.
\item[docs] Generates source documentation (may require the additional
  \verb|haddock| package to be installed)
\end{description}

To compile the program, run \verb|make|.

\subsection{Invocation}
The compiler binary will be put in \verb|dist/build/cmc/cmc|. It
accepts one of two flags, followed by a file path:

\begin{description}
\item[-a] Show the parsed abstract syntax tree.
\item[-c] Check for errors only; either prints syntax errors or ``No
  errors'' if there are none.
\end{description}

If given any other combination of arguments, it will print usage
information to the terminal.


\section{Implementation notes}
\label{implementation}

The compiler is written in Haskell, using the Parsec library to make
parsing easier. In total, the source code weighs in at approximately
300 lines, including blank lines, comments and data structure
declarations (for the abstract syntax tree).




\section{Example syntrax tree}
Given the simple C-Minus program to print the numbers from 0 to 9:

\begin{verse}
\begin{verbatim}
   int main(void)
{
    int a;
    a = 0;
    while(a < 10) {
        output(a);
        a = a + 1;
    }

    return 0;
}
\end{verbatim}
\end{verse}

The compiler will produce the following abstract syntax
tree\footnote{The text has been reformatted to fit on the page.}:

\begin{verse}
\begin{verbatim}
   [Function Int "main" []
      (CompoundStatement [Variable Int "a"]
         [ExpressionStatement
            (AssignmentExpr
               (VariableRef "a")
               (ValueExpr (IntValue 0))),
          IterationStatement
            (ArithmeticExpr Less
               (ValueExpr (VariableRef "a"))
               (ValueExpr (IntValue 10)))
            (CompoundStatement []
               [ExpressionStatement
                  (ValueExpr
                   (FunctionCall "output"
                    [ValueExpr (VariableRef "a")])),
                ExpressionStatement
                  (AssignmentExpr (VariableRef "a")
                     (ArithmeticExpr Add
                        (ValueExpr (VariableRef "a"))
                        (ValueExpr (IntValue 1))))]),
          ValueReturnStatement (ValueExpr (IntValue 0))])]
\end{verbatim}
\end{verse}

This output is Haskell data structures produced by
the parser, pretty-printed for legibility.

\section{Limitations}
Although error detection and reporting is very thorough, no support
has been added for error recovery. That is, once a parsing error has
been encountered, it is printed and the program aborts. This is due to
a limitation of Parsec; it provides no such capability. To add it
would have greatly complicated the parsing rules and syntax tree.


\end{document}
